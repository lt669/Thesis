\documentclass[../../main.tex]{subfiles}

 
\begin{document}
\clearpage
\thispagestyle{empty}
	
	\section*{Abstract}

		 \vspace{5cm}
		 \begin{center}
		 \begin{minipage}{0.7\textwidth}
		% The project described in this document builds upon the Virtual Singing Studio, a loudspeaker based room acoustic simulation system, by extending its functionality to allow a user to move themselves around a virtual space through the use of a large grid of synthetically produced room impulse responses. A number of user tests were conducted in order to obtain data regarding the plausibility of the system. The process of implementing said system was assessed and its plausibility investigated, with results indicating that further work must be done before being able to fully  . Though the system in place provides a framework to build upon, it was found that for a real time application, the system in question encounters too many technical difficulties to be considered accurate or a beneficial tool.

		% A technique for enabling mobility within a Virtual Acoustic Environment using a direct Room Impulse Response rendering method is implemented. A digital room model of Hendrix Hall was created and used to produce a grid of synthetic Room Impulse Responses, using the room acoustic simulation software, Odeon. Max/MSP was used to produce a system that allows a user to move themselves around the virtual version of Hendrix Hall via an iPad whilst.

		 A technique for enabling mobility within a Virtual Acoustic Environment using a direct Room Impulse Response rendering method is implemented as an extension to the Virtual Singing Studio, a room acoustics simulation system. A digital room model of Hendrix Hall was created and used to produce a grid of synthetic Room Impulse Responses, using the room acoustic simulation software, Odeon. By using software Max/MSP, it was possible to produce a interactive system that allows a user to draw themselves a path for which to move around the virtual environment. A number of user tests were conducted to assess the plausibility of the system, which indicated that the density of the Room Impulse Response grid used has less of an impact compared to other factors, such as the rate at which one is moved through the virtual environment.


		 % Though technical difficulties prevented the thorough investigation of the plausibility of the originally designed system, results were obtained 

		 % Though results indicate that a the grid of RIRs used should be separated by a distance of 3m for the system in question, further work is required in before a definitive judgement can be made on the plausibility of a system such as the one implemented.
		 %Though results regarding the plausibility of the implemented system were obtained, it was found that due to inaccuracies in both the system and the impulse responses used, improvements upon the system are required before more definitive judgement on the plausibility of the methods used in implementing the system can be made.

		 % It was found that an distance of 3m between impulse response locations for the implemented system was optimal, however improvement to the system are required before the  A system was produced to accommodate such functionality and the minimum number of impulse responses required for the system to . Though the system implemented provided the functionality intended,  It was found that the system that had been implemented 

		%. The direct room impulse response rendering method was chosen, where a large number of room impulse responses are produced in a grid in an attempt to provide an accurate approximation of the room acoustics for any location within the room by interpolation between a number of RIRs. During the implementation of such a system it was found to be too computationally expensive, thus a compromise was found.
		 \end{minipage}
		 \end{center}
		 \vspace{5mm}

\end{document}
