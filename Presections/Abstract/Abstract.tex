\documentclass[../../main.tex]{subfiles}

 
\begin{document}
\clearpage
\thispagestyle{empty}
	
	\section{Abstract}

			 \vspace{5mm}
		 \begin{center}
		 \begin{minipage}{0.6\textwidth}
		 The MEng project described in this document builds upon the Virtual Singing Studio (VSS), a loudspeaker based room acoustic simulation system, by extending its functionality to allow a user to move themselves around a virtual space through the use of a large grid of room impulse responses. A system was produced to accommodate such functionality and tested. It was found that ... \textbf{FINISH THIS}

		%. The direct room impulse response rendering method was chosen, where a large number of room impulse responses are produced in a grid in an attempt to provide an accurate approximation of the room acoustics for any location within the room by interpolation between a number of RIRs. During the implementation of such a system it was found to be too computationally expensive, thus a compromise was found.
		 \end{minipage}
		 \end{center}
		 \vspace{5mm}

\end{document}
