\documentclass[../../main.tex]{subfiles}

 
\begin{document}
\clearpage
\thispagestyle{empty}
	
	\section{Supporting Material}

	Here is a link to a supporting web page for this document. It contains all of the code, audio samples and videos referred to within this document.

	A supporting web page for the work described in this document can be found \href{http://lt669.github.io}{here}. All code, audio sample and videos mentioned in the text can be viewed/heard by clicking on the file name provided.

	All relevant data produced as a result of this project, including that found on the supporting webpage can be found within the appropriate sub file within the following file structure:

	\begin{forest}
  for tree={
    font=\ttfamily,
    grow'=0,
    child anchor=west,
    parent anchor=south,
    anchor=west,
    calign=first,
    edge path={
      \noexpand\path [draw, \forestoption{edge}]
      (!u.south west) +(7.5pt,0) |- node[fill,inner sep=1.25pt] {} (.child anchor)\forestoption{edge label};
    },
    before typesetting nodes={
      if n=1
        {insert before={[,phantom]}}
        {}
    },
    fit=band,
    before computing xy={l=15pt},
  }
[Dissertation.pdf
  [0) SupportingFiles
    [1) Audio
    	[1.1) Anechoic Recordings]
    	[1.2) Calibrated RIRs]
    	[1.3) Odeon Settings]
    ]
    [2) Matlab]
    [3) MaxPatch
    	[3.1) Javascript files]
    	[3.2) Max patch file]
    ]
    ]
 ]
\end{forest}

% For example, if the text were to read:

% \begin{center}
%  \textit{...the relavant code can be found in file 3.1...} 
% \end{center}

%  the file in question can be found in the \texttt{Javascript files} folder.

\end{document}