\documentclass[../../main.tex]{subfiles}

 \lhead{Implementation: Room Choice}
 
\begin{document}
	\section{Implementation}
	This section describes the steps taken from setting the project objectives to reaching the final implementation of the desired system.

	\subsection{Room Choice}
	\label{roomChoice}

		The first step required before any progress can be made was to find a suitable room for modelling and recording \ac{RIR}'s.

		The following room features were kept in mind when searching for a room to use as part of this project:
		\begin{center}
		\begin{tabular}{r p{12cm}}
			Size: & Large enough to be used as a singing space \\
			Simplicity: & Simple enough architecture to be able to model with the time available \\
			Accessibility: & The room had to be easily accessible to take  multiple measurements to make blue prints and take \ac{RIR} measurements
		\end{tabular}
		\end{center}

		\vspace{5mm}

		The room chosen was Hendrix Hall, a large lecture theatre on the University of York campus. The room contains retractable seating leaving the large space in the centre open and other than two desks at the front of the room, unobstructed. The room is architecturally simple being almost perfectly rectangular with the occasional wall indent. With it being located on the universities campus it could be booked for any time of the week meaning it could be accessed when necessary and for free.

	\subsection{User Test Planning}
	\label{Background:RIRPositions}

		In order to perform the final user tests (test \#1 and test \#2), minimum distance between \ac{RIR}'s had to be decided. Upon finding and measuring Hendrix Hall the maximum number of \ac{RIR}'s that would need to be produced could be calculated given a minimum distance apart. Originally a separation of 0.5m was considered, however upon measuring Hendrix Hall, it was calculated that this would involve measuring 1,920 \ac{RIR} measurement. As it was not known how long producing a bulk of \ac{RIR}'s would taken, it was decided instead to use a 1m separation between \ac{RIR}'s meaning a total of 960 \ac{RIR}'s would need to be calculated.

\end{document}
	