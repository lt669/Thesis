\documentclass[../../main.tex]{subfiles}

 \lhead{Implementation: Room Choice}
 
\begin{document}
	\section{Implementation}
		This section describes the practical work undertaken as part of this project, explaining the choices made and issues raised.

	\subsection{Room Choice}
	\label{roomChoice}

		The first step required before any progress could be made was to find a suitable room for modelling and recording \ac{RIR}'s. The following room features were kept in mind when searching for a room to use as part of this project:
		\begin{center}
		\begin{tabular}{r p{12cm}}
			Size: & Large enough to be used as a singing space. \\
			Simplicity: & Simple enough architecture to be able to model accurately with the time available. \\
			Accessibility: & The room had to be easily accessible for taking multiple measurements and \ac{RIR} recordings.
		\end{tabular}
		\end{center}

		\vspace{5mm}

		Hendrix Hall is a large lecture theatre on the University of York campus and fulfilled the above stated requirements, this was chosen to be used in this project. The room contains retractable seating leaving the large space in the centre open and other than two desks at the front of the room, unobstructed. The room is architecturally simple being almost perfectly rectangular with the occasional wall indent. With it being located on the universities campus it could be booked for any time of the week meaning it could be accessed when necessary and for free.

	\subsection{User Test Planning}
	\label{Background:RIRPositions}

		In order to perform the final user tests (test \#1 and test \#2), a minimum distance between \ac{RIR}'s had to be decided. Upon finding and measuring Hendrix Hall, the maximum number of \ac{RIR}'s that would need to be produced could be calculated for a given minimum distance of \ac{RIR} location separation. Originally a separation of 0.5m was considered, however, upon obtaining the dimensions of Hendrix Hall, it was calculated that this would involve producing 1,920 \ac{RIR}s. As it was not known how long producing a bulk of \ac{RIR}'s would take, it was decided that a 1m separation between \ac{RIR}'s would be sufficient, meaning a total of 960 \ac{RIR}'s would need to be calculated.

		As user test \#3 would require producing a number \ac{RIR} grids of different densities, the \textbf{maximum} distance between \ac{RIR}'s that could be used was calculated. It was decided that a maximum distance for \ac{RIR} separation would be 5m, which would produce a 3x3 grid of \ac{RIR}'s. This meant that a total of 5 grids were to be used in user test \#3, each containing \ac{RIR}'s separated by an extra meter in each from 1m to 5m.

\end{document}
	