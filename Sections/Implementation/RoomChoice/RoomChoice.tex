\documentclass[../../main.tex]{subfiles}

 \lhead{Implementation: Room Choice}
 
\begin{document}
	\section{Implementation}
	This section describes the steps taken from setting the project objectives to reaching the final implementation of the desired system.

	This section describes the practical work undertaken to produce the system previously specified and contains the following sections:

	\subsection{Room Choice}
	\label{roomChoice}

		The first step required before any progress can be made was to find a suitable room for modelling and recording \ac{RIR}'s.

		The following room features were kept in mind when searching for a room to use as part of this project:
		\begin{center}
		\begin{tabular}{r p{12cm}}
			Size: & Large enough to be used as a singing space \\
			Simplicity: & Simple enough architecture to be able to model with the time available \\
			Accessibility: & The room had to be easily accessible to take  multiple measurements to make blue prints and take \ac{RIR} measurements
		\end{tabular}
		\end{center}

		The room chosen was Hendrix Hall, a large lecture theatre on the University of York campus. The room contains retractable seating leaving the large space in the centre open and unobstructed. The room is architecturally simple being almost perfectly rectangular with the occasional wall indent. With it being located on the universities campus it can be booked for any time of the week meaning it can be accessed when necessary and for free.

	\subsection{User Test Planning}
	\label{Background:RIRPositions}

		In order to perform the final user tests (test \#1 and test \#2), minimum distance between \ac{RIR}'s had to be decided.

		It was only after the measurement of Hendrix Hall that this could be decided, as the size of the room determines how many \ac{RIR}'s can be taken


\end{document}
	